\documentclass[a4paper]{article}
\usepackage{vntex}
%\usepackage[english,vietnam]{babel}
%\usepackage[utf8]{inputenc}

%\usepackage[utf8]{inputenc}
%\usepackage[francais]{babel}
\usepackage{a4wide,amssymb,epsfig,latexsym,multicol,array,hhline,fancyhdr}

\usepackage{amsmath}
\usepackage{lastpage}
\usepackage[lined,boxed,commentsnumbered]{algorithm2e}
\usepackage{enumerate}
\usepackage{color}
\usepackage{graphicx}							% Standard graphics package
\usepackage{array}
\usepackage{tabularx, caption}
\usepackage{multirow}
\usepackage{multicol}
\usepackage{rotating}
\usepackage{graphics}
\usepackage{geometry}
\usepackage{setspace}
\usepackage{epsfig}
\usepackage{tikz}
\usetikzlibrary{arrows,snakes,backgrounds}
\usepackage{hyperref}
\hypersetup{urlcolor=blue,linkcolor=black,citecolor=black,colorlinks=true} 
%\usepackage{pstcol} 								% PSTricks with the standard color package

\newtheorem{theorem}{{\bf Định lý}}
\newtheorem{property}{{\bf Tính chất}}
\newtheorem{proposition}{{\bf Mệnh đề}}
\newtheorem{corollary}[proposition]{{\bf Hệ quả}}
\newtheorem{lemma}[proposition]{{\bf Bổ đề}}


%\usepackage{fancyhdr}
\setlength{\headheight}{40pt}
\pagestyle{fancy}
\fancyhead{} % clear all header fields
\fancyhead[L]{
 \begin{tabular}{rl}
    \begin{picture}(25,15)(0,0)
    \put(0,-8){\includegraphics[width=8mm, height=8mm]{hcmut.png}}
    %\put(0,-8){\epsfig{width=10mm,figure=hcmut.eps}}
   \end{picture}&
	%\includegraphics[width=8mm, height=8mm]{hcmut.png} & %
	\begin{tabular}{l}
		\textbf{\bf \ttfamily Trường Đại Học Bách Khoa Tp.Hồ Chí Minh}\\
		\textbf{\bf \ttfamily Khoa Khoa Học và Kỹ Thuật Máy Tính}
	\end{tabular} 	
 \end{tabular}
}
\fancyhead[R]{
	\begin{tabular}{l}
		\tiny \bf \\
		\tiny \bf 
	\end{tabular}  }
\fancyfoot{} % clear all footer fields
\fancyfoot[L]{\scriptsize \ttfamily Bài tập lớn môn Toán Rời Rạc 1 - Niên khóa 2012-2013}
\fancyfoot[R]{\scriptsize \ttfamily Trang {\thepage}/\pageref{LastPage}}
\renewcommand{\headrulewidth}{0.3pt}
\renewcommand{\footrulewidth}{0.3pt}


%%%
\setcounter{secnumdepth}{4}
\setcounter{tocdepth}{3}
\makeatletter
\newcounter {subsubsubsection}[subsubsection]
\renewcommand\thesubsubsubsection{\thesubsubsection .\@alph\c@subsubsubsection}
\newcommand\subsubsubsection{\@startsection{subsubsubsection}{4}{\z@}%
                                     {-3.25ex\@plus -1ex \@minus -.2ex}%
                                     {1.5ex \@plus .2ex}%
                                     {\normalfont\normalsize\bfseries}}
\newcommand*\l@subsubsubsection{\@dottedtocline{3}{10.0em}{4.1em}}
\newcommand*{\subsubsubsectionmark}[1]{}
\makeatother


\begin{document}

\begin{titlepage}
\begin{center}
ĐẠI HỌC QUỐC GIA THÀNH PHỐ HỒ CHÍ MINH \\
TRƯỜNG ĐẠI HỌC BÁCH KHOA \\
KHOA KHOA HỌC - KỸ THUẬT MÁY TÍNH 
\end{center}

\vspace{1cm}

\begin{figure}[h!]
\begin{center}
\includegraphics[width=3cm]{hcmut.png}
\end{center}
\end{figure}

\vspace{1cm}


\begin{center}
\begin{tabular}{c}
\multicolumn{1}{l}{\textbf{{\Large MÔ HÌNH HÓA TOÁN HỌC (MO2011)}}}\\
~~\\
\hline
\\
\multicolumn{1}{l}{\textbf{{\Large Đề bài tập lớn}}}\\
\\
\textbf{{\large \centering {"Đặc tả Smart Contrast bằng Linear Logic"}}}\\
\\
\hline
\end{tabular}
\end{center}

\vspace{1cm}

\begin{table}[h]
\begin{tabular}{rrl}
\hspace{5 cm} & GVHD: & Nguyễn An Khương\\
&     & Huỳnh Tường Nguyên \\
&     & Trần Văn Hoài \\
&     & Lê Hồng Trang \\
&     & Trần Tuấn Anh \\
& SV: & Bùi Bảo Cường- 22102134 \\
& & Trương Hoàng Huy - 1611352 \\
& & Phan Thị Ngọc Ánh - 88471334 \\
& & Nguyễn Thị Thanh Nhật - 88471334 \\
& & Nguyễn Quang Hoàng Lâm - 1611743 \\
\end{tabular}
\end{table}


\end{titlepage}


%\thispagestyle{empty}

\newpage
\tableofcontents
\newpage

Bài báo cáo này trình bày về thống kê và phân tích dữ liệu chiều cao của ca sĩ ở New York Choral Society năm 1979, được chia thành 4 cột lần lượt theo giọng nữ cao, nữ trầm, nam cao và nam trầm.

%%%%%%%%%%%%%%%%%%%%%%%%%%%%%%%%%
\section{Bài toán 1}
\subsection{Giới thiệu về Smart Contrast}
\subsubsection{Lịch sử của Smart Contrast}
\subsubsection{Ứng dụng của Smart Contrast}
\subsection{Linear Logic}
\subsubsection{Lịch sử về Linear Logic}
\subsubsection{Ứng dụng của Linear Logic}
\subsubsection{Ví dụ về đặc tả các Smart Contrast bằng Linear Logic}


%%%%%%%%%%%%%%%%%%%%%%%%%%%%%%%%%
\section{Bài toán 2}
\subsection{Mô tả ngữ cảnh bằng lời}

\subsection{Chuyển ngữ cảnh về các điều khoản}


%%%%%%%%%%%%%%%%%%%%%%%%%%%%%%%%%
\section{Bài toán 3}
\subsection{Đặc tả bài toán 2 bằng Linear Logic}


%%%%%%%%%%%%%%%%%%%%%%%%%%%%%%%%%%%
\section{Bài toán 4}
\subsection{Mã giả cho bài toán 2}


%%%%%%%%%%%%%%%%%%%%%%%%%%%%%%%%%%%
\section{Bài toán 5}
\subsection{Lập trình bằng Solidity}


%%%%%%%%%%%%%%%%%%%%%%%%%%%%%%%%%%%
\begin{thebibliography}{80}





\end{thebibliography}
\end{document}

